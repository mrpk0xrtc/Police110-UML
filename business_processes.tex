\section{خلاصه فرآیند کاری}

\begin{wrapfigure}{l}{1in}
  \begin{center}
    \begin{tikzpicture}[remember picture]
    \node[anchor=north west]
        at (current page.north west) {
        \includesvg[width=14in, height=5in]{images/business_processes}
    };
    \end{tikzpicture}
  \end{center}
\end{wrapfigure}

\hspace{1cm}

می توان فرآیند های درگیر در این سازمان را به دو بخش پشتیبانی و عملیات تقسیم کرد. در بخش پشتیبانی پرسنل دوره میبینند و عملیات های گشت با نیروی انسانی و تجهیزات تأمین میشوند.
\newline
یک عملیات نیاز به منابع انسانی و تجهیزات دارد. سپس با اعزام و تشکیل پرونده ی تحقیقات, گزارشی از پرونده موجود ارائه میشود.
\newline
\vspace{1cm}
\par\noindent
\rl{اختیار نیروی انسانی میتواند از طریق آموزش انجام گیرد.}
\hfill
\lr{\textbf{trains:}}
\newline
\rl{عملیات گشت میتواند درخواست تجهیزات کند.}
\hfill
\lr{\textbf{equips:}}
\newline
\rl{عملیات گشت میتواند درخواست منابع انسانی کند.}
\hfill
\lr{\textbf{staffs:}}
\newline
\rl{عملیات گشت میتواند درخواست اعزام نیرو کند.}
\hfill
\lr{\textbf{calls for:}}
\newline
\rl{مرحله اعزام میتواند منجر به آغاز تحقیقات شود.}
\hfill
\lr{\textbf{assigns:}}
\newline
\rl{یک مرحله تحقیقات میتواند گزارش تولید کند.}
\hfill
\lr{\textbf{generates:}}
