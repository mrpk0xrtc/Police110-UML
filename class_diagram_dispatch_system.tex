\subsection{\lr{Class Diagram - Dispatch System}}

\hspace{1cm}
در این بخش به توضیح فرآیند اعزام نیرو ها به ماموریت ها و حوادث مختلف می پردازیم.
\newline

\begin{wrapfigure}{l}{1.2in}
  \begin{center}
    \begin{tikzpicture}[remember picture]
    \node[anchor=north west]
        at (current page.north west) {
        \includesvg[width=15in, height=6in]{images/class_diagram_dispatch_system}
    };
    \end{tikzpicture}
  \end{center}
\end{wrapfigure}

\hspace{1cm}

\par\noindent
\rlap{\rl{یک ماموریت اعزام می تواند چندین تماس اضطراری دریافت کند.}}
\hfill
\llap{\lr{\textbf{processes:}}}
\newline
\rlap{\rl{یک ماموریت اعزام می تواند مربوط به چندین حادثه باشد.}}
\hfill
\llap{\lr{\textbf{creates:}}}
\newline
\rlap{\rl{یک ماموریت اعزام می تواند شامل چندین واحد باشد.}}
\hfill
\llap{\lr{\textbf{manages:}}}
\newline
\rlap{\rl{هر تماس اضطراری فقط مربوط به یک حادثه است.}}
\hfill
\llap{\lr{\textbf{generates:}}}
\newline
\rlap{\rl{یک حادثه می تواند به چندین واحد سپرده شود.}}
\hfill
\llap{\lr{\textbf{assignedTo:}}}
\newline
\rlap{\rl{هر واحد از یک یا چند افسر تشکیل می شود.}}
\hfill
\llap{\lr{\textbf{consistsOf:}}}
\newline
\rlap{\rl{به هر واحد یک وسیله نقلیه تعلق می گیرد.}}
\hfill
\llap{\lr{\textbf{includes:}}}
\newline
\rlap{\rl{به هر افسر فقط یک وسیله می تواند تعلق گیرد.}}
\hfill
\llap{\lr{\textbf{assignedTo:}}}
\newline
