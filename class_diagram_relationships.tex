\subsection{\lr{Class Diagram - Relationships}}

\hspace{1cm}
در این بخش به توضیح روابط میان کلاس ها و نوع روابط آنها می پردازیم.
\newline

\subsubsection{\lr{Person, Incident, Report}}

\begin{center}
  \begin{tikzpicture}[remember picture]
  \node[anchor=north west]
      at (current page.north west) {
      \includesvg[width=13in, height=4.5in]{images/class_relationships_person}
  };
  \end{tikzpicture}
\end{center}

\hspace{1cm}

\par\noindent
\rl{چندین شخص می توانند در چندین حادثه حضور داشته باشند.}
\hfill
\lr{\textbf{involvedIn:}}
\newline
\rl{یک حادثه می تواند هیچ یا چند متهم داشته باشد.}
\hfill
\lr{\textbf{suspectedIn:}}
\newline
\rl{یک حادثه می تواند هیچ یا چند قربانی داشته باشد.}
\hfill
\lr{\textbf{victimOf:}}
\newline
\rl{هر حادثه می میتواند هیچ یا چند شاهد داشته باشد.}
\hfill
\lr{\textbf{witnesses:}}
\newline
\rl{یک افسر می تواند هر تعداد گزارش بنویسد.}
\hfill
\lr{\textbf{authors:}}
\newline
\rl{یک حادثه به وسیله یک یا چند گزارش توصیف می شود.}
\hfill
\lr{\textbf{documentedBy:}}
\newline

\pagebreak

\subsubsection{\lr{Case}}

\begin{center}
  \begin{tikzpicture}[remember picture]
  \node[anchor=north west]
      at (current page.north west) {
      \includesvg[width=6in, height=4in]{images/class_relationships_case}
  };
  \end{tikzpicture}
\end{center}

\hspace{1cm}

\par\noindent
\rl{هر پرونده شامل هیچ یا چند مدرک است.}
\hfill
\lr{\textbf{contains:}}
\newline
\rl{هر پرونده شامل یک یا چند گزارش است.}
\hfill
\lr{\textbf{includes:}}
\newline
\rl{هر پرونده شامل یک یا چند حادثه است.}
\hfill
\lr{\textbf{comprises:}}
\newline
\rl{هر پرونده شامل هیچ یا چند متهم است.}
\hfill
\lr{\textbf{involves:}}
\newline
\rl{هر پرونده دارای هیچ یا چند شاهد است.}
\hfill
\lr{\textbf{refrences:}}
\newline
