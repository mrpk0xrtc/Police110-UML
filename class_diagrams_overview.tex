\section{\lr{Class Diagrams}}
در این بخش به طراحی کلاس ها می پردازیم. کلاس ها قالب هایی هستند که دارای ویژگی ها و توابعی هستند که میتوانند برای درج و یا حذف ویژگی ها از نمونه ها استفاده شوند و دسته ی دیگر توابع با بخش های دیگر سیستم ارتباط می گیرند.
\subsection{\lr{Class Diagrams Overview}}

\hspace{1cm}

\par\noindent
\rl{یک شخص میتواند در چندین حادثه حضور داشته باشد.}
\hfill
\lr{\textbf{involvedIn:}}
\newline
\rl{یک حادثه میتواند شامل چندین پرونده باشد.}
\hfill
\lr{\textbf{includes:}}
\newline
\rl{یک حادثه میتواند چندین گزارش داشته باشد.}
\hfill
\lr{\textbf{documentedBy:}}
\newline

\vspace{0.5cm}

\begin{wrapfigure}{l}{2.5in}
  \begin{center}
    \begin{tikzpicture}[remember picture]
    \node[anchor=north west]
        at (current page.north west) {
        \includesvg[width=14.5in, height=5in]{images/class_diagrams_overview}
    };
    \end{tikzpicture}
  \end{center}
\end{wrapfigure}

ابتدا یک نمای کلی از اشیاء مورد استفاده در سیستم رسم می کنیم. در ساده ترین حالت چندین نفر و چندین پرونده می توانند درگیر یک حادثه شوند و برای هر حادثه می توان هر تعداد گزارشی تهیه کرد.
\newline\newline

این یک نمای کلی است و تعداد و نوع ویژگی ها می تواند بسته به نیاز سازمان متغیر باشد. توابع مورد نیاز هر شئ در بخش های بعدی بررسی می شود.
