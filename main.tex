\documentclass{article}

\usepackage{hyperref}
\usepackage[toc]{multitoc}
\usepackage{tikz}
\usepackage[inkscapeformat=png,inkscapepath=build]{svg}
\usepackage{xepersian}
\usepackage{enumerate}
\include{config}

\settextfont{Vazirmatn}
\setlatintextfont{FreeSans}
\setlength{\columnseprule}{0.5pt}
\hypersetup{
    colorlinks=true,
    linkcolor=blue,
    filecolor=magenta,
    urlcolor=cyan,
    pdftitle={Police110-UML}
}

\title{{\Huge سیستم مدیریت پلیس 110}}
\author{{\Large \myname}}
\date{{\large \today}}

\begin{document}

\maketitle
\thispagestyle{empty}
\begin{center}
\vspace{0.6cm}
ویرایش \begin{latin}1.0\end{latin}
\vspace{12cm}
\myorg
\end{center}
\pagebreak

\tableofcontents
\pagebreak

\section{مقدمه}
این مقاله با هدف بررسی و مدلسازی سیستم سازمانی پلیس 110 برای توسعه نرم افزار تهیه میشود. سعی میشود با ارائه تعاریف و مدل های UML این امر فراهم گردد.
\newline\newline
توسعه یافته به وسیله \LaTeX و \textsf{\XePersian} و PlantUML
\pagebreak

\subsection{مشکلات}

\begin{itemize}
    \item ناکارآمدی نگهداری پرونده ها به روش سنتی
    \item ارتباطات بین سازمانی ضعیف
    \item اختصاص غیر بهینه منابع
    \item روابط عمومی ضعیف
\end{itemize}

\subsection{اهداف}

\begin{itemize}
    \item خودکار سازی عملیات های پایگاه پلیس
    \item تسهیل مدیریت مدارک
    \item بهبود روابط عمومی پلیس
    \item تقویت بررسی عملکرد مأموران
\end{itemize}

\section{دامنه}

\subsubsection{درون گستره}

\begin{itemize}
    \item گزارش و مدیریت تخلفات و حوادث
    \item مدیریت گشت پلیس
    \item مدیریت اطلاعات و شیفت پرسنل
    \item مدیریت اسناد و مدارک
    \item مدیریت و بازرسی پرونده
    \item پرتال روابط عمومی
\end{itemize}

\subsubsection{برون گستره}

\begin{itemize}
    \item الگوریتم های پیش بینی تخلفات
    \item الگوریتم های تشخیص چهره
    \item عملیات های امنیت ملی
    \item عملیات های بین سازمانی
\end{itemize}

\pagebreak

\begin{tikzpicture}[remember picture, overlay]
    \node[anchor=north west, xshift=1cm, yshift=-2cm]
        at (current page.north west) {
        \includesvg[width=7.5in, height=2.8in]{images/project_scope}
    };
\end{tikzpicture}

\vspace{1cm}

\subsection{ذی‌نفع}

\begin{enumerate}
    \item مأموران پلیس - کاربران اصلی
    \item پایگاه پلیس - مدیر
    \item مأموران بازرسی - کاربران خاص
    \item عموم مردم - کاربران خارجی
    \item رئیس پلیس - کاربران ناظر
    \item پزشکی قانونی - کابران اصلی خارجی
    \item قضات - مصرف کنندگان اطلاعت
\end{enumerate}

\pagebreak

\end{document}
